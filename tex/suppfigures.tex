\documentclass[11pt]{article}

\usepackage[left=1in,right=1in,top=0.5in,bottom=0.7in]{geometry}
\usepackage[scaled]{helvet}
\renewcommand\familydefault{\sfdefault} 
\usepackage[T1]{fontenc}
\usepackage{graphicx}
\usepackage{pdfpages}
\usepackage{fullpage}
\pagestyle{empty} % no page numbers 
\usepackage{pgffor}
\begin{document}

\renewcommand{\figurename}{Supplementary Figure}

\begin{figure}
  \centering
  \includegraphics[width=\textwidth]{suppfig-snps.pdf}
  \caption{SNP coverage across basis traits. \textbf{a} number of SNPs available across each basis study. \textbf{b} number of SNPs in common across varying numbers of basis traits.}
  \label{fig:snpcoverage}
\end{figure}

\begin{figure}
  \centering
  \includegraphics[width=\textwidth]{suppfig-reconstruction-error.png}
  \caption{Mean squared reconstruction error as the number of components used from the principal component decomposition increases from 1 to 14. The error is minimised with 13 or 14 components.}
  \label{fig:recon-error}
\end{figure}

\begin{figure}
  \centering
  \includegraphics[width=\textwidth]{figure2-hclust-rivas-supplementary.png}
  \caption{Hierarchical clustering of basis diseases and their UKBB counterparts in \textbf{a} hard-thresholded, LD-thinned basis constructed using Z scores \textbf{b} hard-thresholded, LD-thinned basis constructed using $\hat\beta$ .  Heatmaps indicate projected $\hat\delta$ for each disease on each component PC1-PC13, with grey indicating 0 (no difference from control), and darker shades of blue or magenta showing departure from controls in one direction or the other. GWAS datasets: T1D = type 1 diabetes, CEL= celiac disease, asthma, MS =multiple sclerosis, UC =ulcerative colitis, CD = Crohn's disease, RA =rheumatoid arthritis, VIT =vitiligo, SLE =systemic lupus erythematosus, PSC=primary sclerosing cholangitis, PBC=primary biliary cholangitis, LADA=latent autoimmune diabetes in adults, IgA\_NEPH= IgA nephropathy. UKBB\_ prefixed diseases correspond to self reported disease status in UK Biobank.}
  \label{sfig:2}
\end{figure}

\foreach \n in {1,...,13}{
\begin{figure}
  \centering
  \includegraphics[width=\textwidth]{suppfig-forest-pc\n.pdf}
  \caption{Forest plot of component PC\n\ showing projected delta and 95\% confidence interval (solid line = FDR $<$ 1\%, dashed line = FDR $\geq$ 1\%). All IMD that are part of a trait group with at least one result significant at FDR $<1$\% are shown, together with any UKBB significant traits. IMD basis disease locations are shown in red.}
  \label{sfig:\n}
\end{figure}
 }
 
 \begin{figure}
  \centering
  \includegraphics[width=\textwidth]{suppfig-consistency.pdf}
  \caption{For each group of traits, we compared the significance of Spearman correlation test of basis component SNP weights with trait $\beta$ (evidence for consistency, see Supplementary Note) with component FDR. We found traits with increasing component significance (smaller FDR) tended to also show more significant Spearman correlations, although the pattern was much weaker for blood cell counts despite more observations with small component FDRs.  We subsequently filtered blood cell counts to count as significant only the outlying trait which was clearly significant by both component FDR and Spearman correlation, corresponding to eosinophil counts on PC13. Datasets are grouped by: blood cell counts,$^{21}$ cytokines,$^{23}$ flow cytometric immune cell counts,$^{22}$ GWAS datasets except UKBB, UKBB (Neale).}
  \label{sfig:2}
\end{figure}

\begin{figure}
  \centering
  \includegraphics[width=\textwidth]{suppfig-ancestry}
  \caption{Traits with GWAS data from different ancestries were identified, and each dataset projected onto the basis. The location of each dataset (delta) is shown for each PC, coloured by reported ancestry.  Solid points had P<0.05 (uncorrected for multiple testing) and lines show their 95\% confidence intervals. Hollow points had P > 0.05.  The square symbol indicates the location of the input data used to learn the basis, and is shown for reference. The dashed vertical line represents the null hypothesis, delta=0. We find that all significant points have the same sign of delta for any given ancestry and PC combination. AFR=African/African American, EAS=East Asian, EUR=European.}
    % \label{fig:ancestry}
\end{figure}

\begin{figure}
   \centering
 \includegraphics[width=0.8\textwidth]{suppfig-proportionality}
   \caption{Comparison of projects of the same traits from datasets with different ancestries.
     We ran parallel analyses of % (a) in psoriatic arthritis, projections from a Spanish and UK (European ancestry) data set are shown with their 95\% confidence intervals for each PC. Projections are very similar, with a constant of proportionality (black line) close to 1 (blue dashed line). The null hypothesis of proportionality is not rejected (p=0.08). (b) in ankylosing spondylitis, there is a more pronounced difference with an attenuation of the projections towards 0 in a Turkish/Iranian ancestry dataset compared to European ancestry, though the two remain proportional (constant of proportionality=0.52, p=0.74).
     % This was supported by a broader comparison of
     two releases of UKBB summary statistics. The Neale compendium focuses on the European subset of UKBB (n approx = 360,000) and GeneAtlas (http://geneatlas.roslin.ed.ac.uk/)(Canela-Xandri, Rawlik, and Tenesa 2018) uses all available UKBB subjects (n approx= 452,000), and a linear mixed model to adjust for population stratification. We tested whether projections for the same trait was proportional across the PCs, and estimated the constant of poportionality for each.  Over 78 traits found in both datasets with similar case counts (within 10\%), (a) GeneAtlas projections were generally proportional to Neale (no significant deviation from proportionality identified) but attenuated (median ratio=0.89) and (b) the additional sample size in GeneAtlas results in generally more significant projections. This suggests the mix of non-European samples leads to an attenuation of signal compared to European-only.  Overall, this suggests that results projecting non-European or GWAS studies on to a European basis may reduce power for the same sample size, but does not lead to invalid results.  }
  \label{sfig:2}
\end{figure}

\begin{figure}
  \centering
  \includegraphics[width=\textwidth]{suppfig-sparsebasis-coefficients.pdf}
  \caption{Distributions of entries in the rotation matrix for each component PC1-PC13}
  \label{sfig:2}
\end{figure}


 
% \begin{figure}
%   \centering
%   \includegraphics[width=\textwidth]{suppfig-mr-astle.pdf}
%   \caption{}
%   \label{sfig:3}
% \end{figure}

% \begin{figure}
%   \centering
%   \includegraphics[width=\textwidth]{suppfig-mr-cytokines.pdf}
%   \caption{}
%   \label{sfig:4}
% \end{figure}

\begin{figure}
  \centering
  \includegraphics[width=\textwidth]{suppfig-sparsesig-qqplots.pdf}
  \caption{QQ plots of p values for driver SNPs on trait-significant components showed a tendency for excess significant results. Points corresponding to FDR $< 1\%$ are highlighted in black, other points in gray. The solid line represents $y=x$.}
  \label{sfig:4}
\end{figure}


\end{document}

%%% Local Variables:
%%% mode: latex
%%% TeX-master: t
%%% End:
