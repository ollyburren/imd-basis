\documentclass[11pt]{article}

\usepackage{helvet}
\usepackage{graphicx}
\usepackage{pdfpages}
\usepackage{fullpage}
\pagestyle{empty} % no page numbers 
\usepackage{pgffor}
\begin{document}

\renewcommand{\figurename}{Supplementary Figure}

\begin{figure}
  \centering
  \includegraphics[width=\textwidth]{suppfig-sparsebasis-coefficients.pdf}
  \caption{Distributions of entries in the rotation matrix for each component PC1-PC13}
  \label{sfig:2}
\end{figure}

 \begin{figure}
  \centering
  \includegraphics[width=\textwidth]{suppfig-ukbb-sig-by-imd}
  \caption{IMD are enriched amongst significant (FDR $<$ 1\%) UKBB traits, with 63\% of IMD even with few cases and $<3\%$ of non-IMD traits significant. x axis is truncated at FDR=$10^{{-6}}$ for display.}
  \label{sfig:2}
\end{figure}

\foreach \n in {1,...,13}{
\begin{figure}
  \centering
  \includegraphics[width=\textwidth]{suppfig-forest-pc\n.pdf}
  \caption{Forest plot of component PC\n\ showing projected delta and 95\% confidence interval (solid line = FDR $<$ 1\%, dashed line = FDR $\geq$ 1\%). All IMD that are part of a trait group with at least one result significant at FDR $<1\%$ are shown, together with any UKBB significant traits. IMD basis disease locations are shown in red.}
  \label{sfig:\n}
\end{figure}
 }


 
 \begin{figure}
   \centering
   \includegraphics{suppfig-proportionality}
   \caption{Comparison of projects of the same traits from datasets with different ancestries.
     (a) in psoriatic arthritis, projections from a Spanish and UK (European ancestry) data set are shown with their 95\% confidence intervals for each PC. Projections are very similar, with a constant of proportionality (black line) close to 1 (blue dashed line). The null hypothesis of proportionality is not rejected (p=0.08). (b) in ankylosing spondylitis, there is a more pronounced difference with an attenuation of the projections towards 0 in a Turkish/Iranian ancestry dataset compared to European ancestry, though the two remain proportional (p=0.74).
     In UKBB, Neale analysed only a white European subset while GeneAtlas includes additional non-European cases.  Over 78 traits found in both datasets with similar case counts (within 10\%), the additional sample size in GeneAtlas results in (c) generally more significant projections while (d) 
the constant of proportionality (GeneAtlas/Neale) is centred on 0.9.}
  \label{sfig:2}
\end{figure}

 
 \begin{figure}
  \centering
  \includegraphics[width=\textwidth]{suppfig-consistency.pdf}
  \caption{For each group of traits, we compared the significance of Spearman correlation test of basis component SNP weights with trait $\beta$ (evidence for consistency, see Supplementary Note) with component FDR. We found traits with increasing component significance (smaller FDR) tended to also show more significant Spearman correlations, although the pattern was much weaker for blood cell counts despite more observations with small component FDRs.  We subsequently filtered blood cell counts to count as significant only the outlying trait which was clearly significant by both component FDR and Spearman correlation, corresponding to eosinophil counts on PC13. Datasets are grouped by: blood cell counts,$^{21}$ cytokines,$^{23}$ flow cytometric immune cell counts,$^{22}$ GWAS datasets except UKBB, UKBB (Neale).}
  \label{sfig:2}
\end{figure}

% \begin{figure}
%   \centering
%   \includegraphics[width=\textwidth]{suppfig-mr-astle.pdf}
%   \caption{}
%   \label{sfig:3}
% \end{figure}

% \begin{figure}
%   \centering
%   \includegraphics[width=\textwidth]{suppfig-mr-cytokines.pdf}
%   \caption{}
%   \label{sfig:4}
% \end{figure}

\begin{figure}
  \centering
  \includegraphics[width=\textwidth]{suppfig-sparsesig-qqplots.pdf}
  \caption{QQ plots of p values for driver SNPs on trait-significant components showed a tendency for excess significant results. Points corresponding to FDR $< 1\%$ are highlighted in black, other points in gray. The solid line represents $y=x$.}
  \label{sfig:4}
\end{figure}


\begin{figure}
  \centering
  \includegraphics[width=\textwidth]{suppfig-reconstruction-error.png}
  \caption{Mean squared reconstruction error as the number of components used from the principal component decomposition increases from 1 to 14. The error is minimised with 13 or 14 components.}
  \label{fig:recon-error}
\end{figure}

\end{document}

%%% Local Variables:
%%% mode: latex
%%% TeX-master: t
%%% End:
